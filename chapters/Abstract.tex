\chapter*{Abstract}

Most of the parameters governing the neutrino oscillations have been measured in the last decades. One of the remaining unknowns is the question of the Neutrino Mass Hierarchy. Answering this question is one of the main missions of the KM3NeT/ORCA detector. One of the critical steps in the NMH measurement is the identification of the flavour of the detected neutrinos. If the current tools used to discriminate between the different flavours -- a machine learning algorithm called Random Decision Forest -- performs well at the high-end of the energy range seen by KM3NeT/ORCA, the performance achieved at the lower edge (1--10 GeV) is relatively poor. Besides, this algorithm use brute force to classify the events by using all of the near two hundred variables reconstructed from the raw data of the detector. The goal of this study is to improve the particle identification at these energies by using a different kind of machine learning algorithm -- a Neural Network -- and to reduce the number of variables necessary to the classification. After introducing the background and tools necessary to the understanding of the study, the performance of the neural network built is compared to the one of the Random Decision Forest already in use. It appears that both objectives are met by the new neural network.

{\let\clearpage\relax \chapter*{Résumé}}

La plupart des paramètres gouvernant les oscillations des neutrinos ont été mesurés durant les dernières décennies. Une question ouverte qui subsiste est la question de la Hiérarchie de Masse des Neutrinos. Celle-ci est une des missions principales du détecteur KM3NeT/ORCA. Une des étapes critiques à la détermination de cette hiérarchie est l'identification de la saveur des neutrinos détectés. Si l'outil utilisé aujourd'hui, un algorithme d'apprentissage automatique appelé Forêt d'Arbres Décisionnels, montre de bonnes performances dans la partie haute du spectre d'énergie des neutrinos détectés par KM3NeT/ORCA, la partie inférieure de ce spectre (1--10 GeV) voit de plutôt faibles performances. En plus de cela, cet algorithme utilise la force brute pour discriminer les évènements en utilisant toutes les quelques deux-cent variables reconstruites à partir des données brutes du détecteur. L'objectif de cette étude est d'améliorer les capacités de discrimination aux basses énergies en utilisant un autre type d'algorithme d'apprentissage automatique, un réseau de neuronnes, et de réduire le nombre de variables nécessaires à la classification. Après avoir introduit les notions et outils nécessaires à la compréhension de cette étude, les performances du réseau de neuronnes construit sont comparées à celles de la Forêt d'Arbres Décisionnels utilisée jusque là. Il apparaît que les deux objectifs sont atteints par le nouveau réseau de neuronnes.
